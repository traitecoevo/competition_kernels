\documentclass[10pt]{article}
\newcommand{\ud}{\ensuremath{\mathrm{d}}}
\usepackage{amsmath}
\usepackage{minionpro}
\newcommand{\sign}{\mathop{\mathrm{sign}}\nolimits}

\begin{document}

The \textit{per-capita} growth rate at invasion is

\begin{equation}
  \label{eq:v1}
  \frac{\ud N(x_I)}{\ud t} = r_I\left(K(x_I) - \alpha(x_I, x_R) N_R\right)
\end{equation}

or

\begin{equation}
  \label{eq:v2}
  \frac{\ud N(x_I)}{\ud t} =
  r_I\left(1 - \frac{\alpha(x_I, x_R) N_R}{K(x_I)}\right)
\end{equation}

Rearranging to solve for $\alpha$:

\begin{equation}
  \label{eq:a1}
  \alpha(x_I, x_R) = \frac{1}{N_R}
  \left(K(x_I) - \frac{1}{r_I}\frac{\ud N(x_I)}{\ud t}\right)
\end{equation}

\begin{equation}
  \label{eq:a2}
  \alpha(x_I, x_R) = \frac{K(x_I)}{N_R}
  \left(1 - \frac{1}{r_I}\frac{\ud N(x_I)}{\ud t}\right)
\end{equation}

These are obviously different;

\begin{equation*}
  \alpha_1(x_I, x_R) - \alpha_2(x_I, x_R) =
  \frac{\ud N(x_I)}{\ud t} \frac{K(x_I) - 1}{N_r r_I}
\end{equation*}

With the second form,
\begin{equation*}
  \sign(\alpha) = \sign\left(1 - \frac{w_I}{r_I}\right)
\end{equation*}
so, $\alpha < 0$ (i.e., facilitation) when $w_I > r_I$, which makes
intuitive sense to me.

With the first form
\begin{equation*}
  \sign(\alpha) = \sign\left(K(x_I) - \frac{w_I}{r_I}\right)
\end{equation*}
so facilitation happens when $K(x_I) r_I < w_I$.  When $K = 1$ (as in
Leimar, I think) this does not really matter.  But it seems a real
problem for models where $K$ actually varies with traits.

\section*{Multiple species}

There is an issue when multiple species are involved though.  For two
residents with traits $x_A$ and $x_B$ we have per-capita fitness

\begin{equation*}
  w_I = r_I\left(1 - \frac{\alpha_{x_I, x_A} N_A + \alpha_{x_I, x_B} N_B}{K(x_I)}\right)
\end{equation*}
Rearranging:
\begin{equation*}
  \alpha_{x_I, x_A} N_A + \alpha_{x_I, x_B} N_B =
  K(x_I)\left(1 - \frac{w_I}{r_I}\right)
\end{equation*}
So we have one equation and two unknowns; there is no way of
estimating both $\alpha$ coefficients from the single measurement of
fitness.

We can vary the fitness of one species though and get extra
information perhaps.  So if we computed fitness at $N_a + \epsilon$ we'd
get one extra equation and one extra unknown.  Taking this idea
further we can compute the derivative of $w_I$ with respect to ${N_a,
  N_b}$ (i.e., the Jacobian of $w_I$) and end up with two more
equations and two more unknowns.  For example:

\begin{equation*}
  \frac{\ud}{\ud N_j} = - \left(\frac{r_I}{K(x_I)} \alpha_{x_I, x_j}
    + \frac{r_I N_j}{K_{x_I}}
    \frac{\partial}{\partial N_j}\left[ \alpha_{x_I, x_j} \right]\right)
\end{equation*}

This does seem gettable though.  I think I'm just missing something.

\end{document}
