\documentclass[12pt,a4paper]{article}
\usepackage{minionpro}
\usepackage{amsmath}
\usepackage{mathtools}
\newcommand{\ud}{\mathrm{d}}
\setcounter{secnumdepth}{0}

\begin{document}

\section{Logistic growth}

This is a classic analysis and nice and straightforward to think
about.  Everything else will build off this.  The continuous time
logistic equation is
%
\begin{equation}
  \label{eq:logistic}
  \frac{\ud N}{\ud t} = r N\left(1 - \frac{N}{K}\right)
\end{equation}
%
where $N$ is the density of the species, $r$ is the instantaneous
per-capita growth rate at zero denssity, and $K$ is the carrying
capacity.  There are two equilibria: $\hat N = 0$ and $\hat N = K$.

Taking the derivative of equation (\ref{eq:logistic}) with respect to
$N$ gives

\begin{equation}
  \label{eq:logistic-derivative}
  \frac{\ud}{\ud N}\left[\frac{\ud N}{\ud t}\right]
   = r - \frac{2 r N}{K}.
\end{equation}
%
When $\hat N = 0$, we have 
\begin{equation}
  \label{eq:logistic-derivative-trivial}
  \frac{\ud}{\ud N}\left[\frac{\ud N}{\ud t}\right]
   = r
\end{equation}
which is unstable for $r > 0$.  For the nontrivial equilubrium, $\hat
N = K$, we have
\begin{equation}
  \label{eq:logistic-derivative-nontrivial}
  \frac{\ud}{\ud N}\left[\frac{\ud N}{\ud t}\right]
   = -r
\end{equation}
which is stable for $r > 0$.

\section{Intraspecific-competition}

We can generalise equation (\ref{eq:logistic}) to include the effect
of intraspecific competition.  

\begin{equation}
  \label{eq:logistic-competition}
  \frac{\ud N}{\ud t} = r N\left(1 - \frac{N\alpha(N)}{K}\right)
\end{equation}

where $\alpha$ is a competition coefficient.  We still have an
equilibrium at $\hat N = 0$ but there are potentially more nontrivial
equilibria --- wherever where $N = K / \alpha(N)$.  Because we can
choose both $K$ and $\alpha$ we can define $\alpha(K) = 1$ and then we
retain an equilirium at $\hat N = K$.

Taking the derivative of equation (\ref{eq:logistic-competition}) with
respect to $N$:
\begin{equation}
  \label{eq:logistic-competition-derivative}
  \frac{\ud}{\ud N}\left[\frac{\ud N}{\ud t}\right]
   = r - \frac{2 r N\alpha(N)}{K} - \frac{rN^2}{K}\alpha^\prime(N)
\end{equation}
where $\alpha^\prime(N) = \ud \alpha/\ud N$.

The conditions at the trivial equilibrium don't change because when $N
= 0$, equation (\ref{eq:logistic-competition-derivative}) reduces to
equation (\ref{eq:logistic-derivative-trivial}).  The nontrivial
equilibrium changes:
\begin{equation}
  \label{eq:logistic-competition-derivative-nontrivial}
  \frac{\ud}{\ud N}\left[\frac{\ud N}{\ud t}\right]
   = -r - rK\alpha^\prime(N) = -r(1 + K\alpha^\prime(N)).
\end{equation}
So the nontrivial equilibrium is stable if $r > 0$ and $K\alpha^\prime
> -1$ (other conditions apply for $r < 0$, but that's less
biologically relevant/interesting: we'd be stable there for $r < 0$
and $K\alpha^\prime < -1$).  Because $K \geq 0$, the sign of
$K\alpha^\prime$ is determined by $\alpha^\prime$.  It is negative
when the per-capita competitive effect decreases with increasing
density.  If it does de-stabilise the nontrivial equilibrium it must
open up another equilibrium between 0 and $K$ though, because the
trivial equilibrium is unconditionally unstable.

The nice thing about this approach is we can partition the stability
analysis into a density independent part and a density dependent
part.  What's not clear is if this will generally be easiest to work
with multiplicatively:
\begin{equation}
  \label{eq:logistic-comptition-partition-multiplicative}
  \widetilde \lambda =
  \overbrace{-r}^{\mathclap{\text{density-independent}}}
  \times
  \underbrace{(1 + K\alpha^\prime(N))}_{\text{density-dependent}}
\end{equation}
or additively:
\begin{equation}
  \label{eq:logistic-comptition-partition-multiplicative}
  \widetilde \lambda =
  \overbrace{-r}^{\mathclap{\text{density-independent}}}
  +
  \underbrace{-rK\alpha^\prime(N)}_{\text{density-dependent}}
\end{equation}

\end{document}

%%% Local Variables: 
%%% mode: latex
%%% TeX-PDF-mode: t
%%% End: 
