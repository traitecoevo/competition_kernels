\documentclass[a4paper,11pt]{article}
\usepackage[osf]{mathpazo}
\usepackage{ms}
\usepackage{natbib}
\usepackage{graphicx}
\usepackage{caption}

% Allow referencing into the supporting information, once that exists.
\IfFileExists{./competition-kernels-supporting.tex}{%
  \usepackage{xr}%
  \externaldocument{competition-kernels-supporting}}{}

\title{Something about competition kernels}
\author{}
\date{}
\affiliation{}
\runninghead{}
\keywords{}

\newcommand{\ud}{\ensuremath{\mathrm{d}}}
\newcommand{\sign}{\mathop{\mathrm{sign}}\nolimits}

\begin{document}

% Disable words breaking over lines for final submission:
% \raggedright
% \pagestyle{empty}

\mstitlepage
\parindent=1.5em
\addtolength{\parskip}{.3em}

% \begin{abstract}
% \end{abstract}

\section{Introduction, take2}

Competition kernels are widely used in both models and arguments.
They represent the fitness landscape drawdown.
\begin{itemize}
\item example usage:
  \begin{itemize}
  \item limiting similarity \citet{MacArthur-1967}
  \item coexistence
  \item characte displacement \citep[e.g.,][]{Taper-1985, Case-2000,
      Goldberg-2006}
  \item game theory \citep[e.g.,][]{Brown-1987-140, Brown-1987}
  \item disruptive selection and adaptive dynamics \citep{Dieckmann-1999}
  \end{itemize}
\item Species packing remains a hugely influential concept: the
  niche is usually reprensented as a Gaussian disstribution in one
  dimension (e.g., Ricklefs' textbooks over the years: p 602 in
  1999, v4).
\end{itemize}

Competition kernels, and models of competition, are largely unchanged
from their original formulation in the 1960s.
\begin{itemize}
\item Based on Lotka-Volterra equations (\citealt{MacArthur-1967}
  references Volterra 1926), with a function for competition in terms
  of a pair of species traits; the main fomulation due to
  \citet{MacArthur-1967}
\item These models have been criticised from the outset:
\begin{itemize}
\item \citet{May-1972}
\item \citet{Neill-1974}: competition functions not meaninful
  properties of species pairs, are density and context dependent.
\item \citet{Abrams-1975}: sensitive to details about how resources
  are used.
\item \citet{Abrams-1980}: implicit assumptions likely to be violated
  leading to density dependent competition
\end{itemize}
\item Surprisingly, most of the criticism of shape focusses on the
  ease with which \emph{continuous coexistence} is possible, rather
  than focussing on what shapes are meaningful: we have very little
  information on the likely shapes in nature aside from the assertion
  that a Gaussian is a likely reasonable low-order approximation.
\end{itemize}

There is a shift in the 1980s from modelling limiting similarity in
terms of resources to modelling trait divergence/convergence and
disruptive selection.
\begin{itemize}
\item Begins with \citet{Taper-1985} and \citet{Brown-1987-140}; these
  make explicit a direct 1:1 trait:resource mapping implied by the
  previous models.
\item Continued as the models were used in adaptive dynamics
  \citep[e.g.,][]{Dieckmann-1999} and then picked up by physicists
  \citep[e.g.,][]{Leimar-2013}
\item For vertebrate animals, this connection has always been fairly
  defendable for the sorts of traits people think about (e.g., finch
  beak size --- Lack), but for plants the resource axis is hard: most
  plants would rather more light.  Shifting to traits with some
  complicated nonlinear map between traits and resources offers a way
  forward, though it makes the links to resources and underlying
  process muddier.
\end{itemize}

At the same time, other strands of modelling aimed to link species
traits, competition and coexistence.
\begin{itemize}
\item R* \citep{Tilman-1980,Tilman-1982} aimed to link resource
  dynamics to competition via species competing for different pools of
  resources.  It contains many of the same ingredients as the
  \citet{MacArthur-1967} model, but with slower replenishment of
  resources.
\item Sessile dynamics to Neutral theory
\item Competition/colonisation models.  Incl lottery models?
\end{itemize}

If we understood competition kernels with respect to traits, we could
\begin{itemize}
\item Measure real competition kernels in nature
\item Predict trait-mixtures better than species mixtures.  Is there a
  signature of competition in extant trait-mixtures?  What would it
  be?
\item Measuring competition along trait gradients in the field likely
  intractable, but possible in models where competition emerges from
  first principle ingredients about biological processes that we think
  are important.
\item Basing an entire field on competition kernels where we don't
  even know their shape seems like a house of cards.
\item We aim to investigate the shape and behaviour of competition
  kernels in mechanstic models of growth and reproduction.
\end{itemize}

\section{Introduction}

\begin{itemize}
\item Origin of competition kernels and the literature that has spread
  from them, especially limiting similarity and their use in models
  where we need species to compete to generate frequency dependent
  fitness.
\item Evidence for shape of competition kernels in nature, motivating
  use of modelling.
\item Identified issues with assumptions about kernels: especially
  density dependence, trait dependence (e.g. width of the kernel as a
  trait), Gaussian pathalogical case.
\item Ideas of explicit competition kernels and implicit competition
\end{itemize}

% Yeah, so this section entirely lacks references.  It's adapted from
% what I said during my Evolution talk.  I think all these points are
% defendable though some are doubtless a lot more subtle than as
% presented here.  That is especially the case for Gaussian
% competition as those in the know avoid Gaussian kernels, and I think
% even some of the very early competition work did too.
Interspecific competition generates frequency dependent selection that
is probably responsible for mantaining much of species diversity, and
disruptive selection that generates phenotypic selection.  Despite its
wide potential importance, the way we model competition is usually
very simple: typically species are assumed to compete most strongly
with themselves and less strongly with things very different from
themselves -- something like a Gaussian shape of competition intensity
with respect to traits (e.g., examples).
% It seems likely that a figure here will help, but this is likely
% pretty old-hat to most people.
This may lead to so-called ``limiting similarity''; species are spaced
along a trait axis, kept apart or excluded from the community due to
intense competition between similar species.
% Lack 1947 is the canonical empirical example here.
This approach to modelling competition can be motivated by imagining
that species use resources that are arranged along some axis
(e.g. seed size) and that some trait (e.g. beak size) affects how
efficiently a species can handle and use the resources.

\begin{itemize}
\item These ideas root back to \citet{MacArthur-1967}, and often the
  modelling approach is essentially unchanged.
\item Do these ideas have stronger roots in animal ecology than plant
  ecology?  Are there fewer resources?  (there is a paper, possibly
  Tilman, that argues that plants \emph{don't} have fewer resources
  than animals).  In any case, it's not clear how most species traits
  map on to resource use directly.  Perhaps contrast distributions
  like \citet{Hermoyian-2002} with distribution of plant traits like
  LMA with essentially continuous distributions.
  % This is still a bit weak: need to be clearer here about the sort
  % of questions we might want to ask.
\item It's therefore hard to bridge between abstract models of
  competition that inform most theory with the sort of traits and
  species data that we have for plant communities.
\end{itemize}

Aims

\begin{itemize}
\item What do competition kernels look like?  Are they symmetric?
\item Are kernels constant across trait space, and as such important
  properties of \emph{traits} rather than species-specific properties
  that are hard to generalise?
\item Are kernels constant with species density?  Or do we need to
  know the densities of all actor species to predict what will happen?
\end{itemize}

\section{Background}

% Getting this bit right is going to be hard but important.  It's also
% destined for the appenndix probably.
A wrinkle: there are two forms of modified Lotka Volterra equations in
the literature

\begin{subequations}
  \label{eq:LV}
  \begin{align}
    \frac{\ud N_i}{\ud t} =& r_i\left(K_i  - \sum_j\alpha(i, j) N_j\right)\\
    \frac{\ud N_i}{\ud t} =&
    r_i\left(1 - \sum_j \frac{\alpha(i, j) N_j}{K_i}\right)
  \end{align}
\end{subequations}

these differ in whether the competition function is scaled by the
species' carrying capacity or not.  For scalar $\alpha$ and $K$
parameters, this difference is a matter of book-keeping, but when
$\alpha$ and $K$ are functions of some trait $x$ difference becomes
important.  By convention, $\alpha(i, i) \equiv 1$.

\section{Methods}

% Most of the methods section will be describing the models that we
% use.  Ideally we'll keep this nice and short, and any actual
% elaboration will be put in the supporting material and in the code.
% This is particularly the case for tree, but hopefully we can mostly
% point at Falster-2011 for that?
%
% In the first paragraph, I guess we describe the entire approach as
% well as possible.
We use two models that feature competition emerging from the model.
In both, competition emerges due to species competing for the same
resources (as in the Lotka-Volterra equations), and species traits
affect how species compete for resources.  We compute the quantities
used in the Lotka-Volterra equations (growth rate, carrying capacity
and per-capita rate of population increase) and from that infer the
values of competition coefficients.
% Make sure to hit all these points in the intro:
We can do this over a range of trait values and investigate things
about these kernels: in particular if they have Gaussian shape, if
they are symmetrical and if they are density independent.

% This section will *all* move into an appendix and we'll present the
% equations we need: probably the equation that relates alpha to N, r,
% K, w.
It is not possible to infer all competition coefficients in the
general case as there are too many unknowns to infer.  Instead we
focus on the special case where there is a single resident type and a
mutant is invading from negligible density.  With $x_I$ and $x_R$ as
the traits of indvading and resident types, then the per-capita rate
of increase in the mutant density is

\begin{subequations}
  \label{eq:LVi}
  \begin{align}
    \frac{\ud N(x_I)}{\ud t} &=
    r(x_I)\left(K(x_I) - \alpha(x_I, x_R) N_R\right)\\
    \frac{\ud N(x_I)}{\ud t} &=
    r(x_I)\left(1 - \frac{\alpha(x_I, x_R) N_R}{K(x_I)}\right)
  \end{align}
\end{subequations}

(with the two forms being related to equation (\ref{eq:LV}))

Rearranging equation (\ref{eq:LVi}) to solve for $\alpha$ gives

\begin{subequations}
  \begin{align}
    \alpha(x_I, x_R) &= \frac{1}{N_R}
    \left(K(x_I) - \frac{1}{r(x_I)}\frac{\ud N(x_I)}{\ud t}\right)\\
    \alpha(x_I, x_R) &= \frac{K(x_I)}{N_R}
    \left(1 - \frac{1}{r(x_I)}\frac{\ud N(x_I)}{\ud t}\right)
  \end{align}
\end{subequations}

These are obviously different;
\begin{equation*}
  \alpha_1(x_I, x_R) - \alpha_2(x_I, x_R) =
  \frac{\ud N(x_I)}{\ud t} \frac{K(x_I) - 1}{N_R r(x_I)}
\end{equation*}

So:

\begin{align*}
  \sign(\alpha_1) =& \sign\left(K(x_I) - \frac{w_I}{r(x_I)}\right)\\
  \sign(\alpha_2) =& \sign\left(1 - \frac{w_I}{r(x_I)}\right)
\end{align*}

Therefore with the second form facilitation ($\alpha < 0$) happens
facilitation happens when $w_I > r(x_I)$, which makes sense.  In
contrast with the first form we get facilitation when $K(x_I) r(x_I) <
w_I$, which does not an intuitive appeal unless $K(x) = 1$ (as in
Leimar, I think).  But it seems a real problem for models where $K$
actually varies with traits.

To show that we have a problem with two species, consider the case
where we have species $A$ and $B$, both at nontrivial density.  Then
the per-capita fitness of an invading type $I$ is

\begin{equation*}
  w_I = r(x_I)\left(1 - \frac{\alpha(x_I, x_A) N_A + \alpha(x_I, x_B) N_B}{K(x_I)}\right)
\end{equation*}

Rearranging:
\begin{equation*}
  \alpha_{x_I, x_A} N_A + \alpha_{x_I, x_B} N_B =
  K(x_I)\left(1 - \frac{w_I}{r(x_I)}\right)
\end{equation*}
So we have one equation and two unknowns; there is no way of
estimating both $\alpha$ coefficients from the single measurement of
fitness.

% R*
\subsection{Tilman's R star model}
\begin{itemize}
\item As originally formulated, R* does not include traits
\item We adapt a formulation (due to whom?) where a ``trait'' is a
  measure of specialisation/generalisation.
\end{itemize}

\subsection{TREE}
\begin{itemize}
\item Originally described by \citet{Falster-2011}
\item Species compete for light in a stand
\item Average light environment constructed, averaged over the
  expected disturbance regime of a patch in an infinite metapopulation
  with island model dispersal.
\item Things we need for the model:
  \begin{itemize}
  \item $N$ --- seed rain
  \item $r(x)$ --- this is the rate of growth in an empty environment
    (i.e., mutant fitness in the empty environment)
  \item $K(x)$ --- this is maximum population density (equilibrium
    seed rain in a monoculture)
  \end{itemize}
\end{itemize}

In a way, we're arguing to turn the way that we've looked at
competition around: rather than asking ``given we know species
coexist, what do we have to do to competition kernels'', we're
saying: ``given we have data on species traits, can we just look at
the shapes that competition functions actually take and use that to
guide future modelling''.

\bibliographystyle{amnat}
\bibliography{references}

\end{document}

%%% Local Variables:
%%% mode: latex
%%% TeX-master: t
%%% TeX-PDF-mode: t
%%% End:
