\documentclass[12pt]{beamer}

% Turn off the ugly navigation symbols.
\setbeamertemplate{navigation symbols}{}

% Margins and whatnot.
\setbeamersize{text margin left=2em,text margin right=2em}

\usepackage{fontspec}
\defaultfontfeatures{Mapping=tex-text} % enable -- / --- / `` / ''
\usefonttheme{serif}
\setmainfont{Helvetica}
\setsansfont{Oswald}
\setmonofont{Courier}

\definecolor{verydarkgrey}{HTML}{001112}
\definecolor{darkblue}{HTML}{232747}
\definecolor{grey}{HTML}{777777}

\setbeamerfont{frametitle}{size=\Large}
\setbeamercolor{frametitle}{fg=black}
\setbeamercolor{titleline}{bg=black}
\setbeamercolor{title}{fg=black}
\setbeamercolor*{date in head/foot}{fg=grey}
\setbeamercolor{item}{fg=black}
\setbeamerfont{item}{series=\bfseries}
% Never used, but this is nice.
\setbeamertemplate{itemize subitem}{--}
\setbeamerfont{itemize/enumerate subbody}{size=\small}

% \useoutertheme{new}

\makeatletter
\usepackage{dashrule}
\defbeamertemplate*{frametitle}{mine}[1][left]
{
  % Increase width of title box
  \@tempdima=\textwidth%
  \advance\@tempdima by\beamer@leftmargin%
  \advance\@tempdima by\beamer@rightmargin%
  %
  \begin{beamercolorbox}[sep=0.3cm,#1,wd=\the\@tempdima]{frametitle}
    \usebeamerfont{frametitle}%
    \vbox{}\vskip-1ex\vspace{.5ex}
    \strut\hspace{2ex}\sf\insertframetitle\strut\par%
    \vskip-1.5ex%
    \hspace{\fill}%
    %\hdashrule{\textwidth}{.075ex}{.075ex .2ex}%
    \color{black}{\rule{\textwidth}{.075ex}}%
    \hspace{\fill}%
    \par\nointerlineskip \vspace{\baselineskip}
  \end{beamercolorbox}%
  \nointerlineskip%
  \vspace{-3.5ex}
}

\defbeamertemplate*{footline}{}
{
  \hbox{%
  \begin{beamercolorbox}[wd=.8\paperwidth]{section in head/foot}%
  \end{beamercolorbox}%
  \begin{beamercolorbox}[wd=.2\paperwidth,ht=2.25ex,dp=1ex,right]{date in head/foot}%
    \insertframenumber{} / \inserttotalframenumber\hspace*{2ex} 
  \end{beamercolorbox}}%
  \vskip0pt%
}
\makeatother


\usepackage{fancyvrb}
\usepackage{fancyvrb}
\usepackage{xcolor}
\usepackage{relsize}
\newcommand{\code}{\texttt}
\usepackage{url}
\newcommand{\Rstar}{R$^*$}
\newcommand{\ud}{\mathrm{d}}

\usepackage[overlay,absolute]{textpos}
\newcommand\FrameText[1]{%
  \begin{textblock*}{\paperwidth}(0pt,0.95\textheight)
    \raggedleft #1\hspace{.5em}
  \end{textblock*}}

\begin{document}

% TODO: Need a nice title slide.  Probably just go with a picture and
% the text in a box.  Simple and relatively easy.  Picture could just
% be a tropical forest shot or something, or a slide split between a
% tropical forest and a monoculture of some sort, which would lead
% into the next slide nicely.
\begin{frame}
  \thispagestyle{empty}
  \begin{center}
    % \sf
    {\Huge Competition Kernels \\\& Coexistance}\\[2ex]
    % {\footnotesize {}}\\[1.5em]
    {\large Rich FitzJohn, Daniel Falster,\\[1ex]
    Georges Kunstler, Mark Westoby}
  \vfill
  Macquarie University
  \end{center}
\end{frame}

\begin{frame}
  % This is a fairly abrupt start, but it's probably important for
  % keeping the talk to time.
  %
  % Interspecific competition as a function of species traits makes it
  % into many ecological and evolutionary models: models of
  % coexistance, character evolution, etc.
  % 
  % The way we model it is very simple: usually we imagine species
  % compete most strongly with themselves and less strongly with
  % things very different from themselves -- practically we model that
  % as a gaussian function in trait space.
  % 
  % This leads to so-called limiting similarity, where species are
  % spaced along a trait axis.
  %
  % This is usually motivated by imagining that species use resources
  % along a correlated axis with some optimium.
  %
  % TODO: Swap out leaf size here for finch size
  \includegraphics<1>[width=\textwidth]{figs/gaussian_competition_3}
  \includegraphics<2>[width=\textwidth]{figs/gaussian_competition_4}
\end{frame}

\begin{frame}
  % These models are old and come from a mix of Lotka-Volterra models
  % and logistic growth and are basically unchanged since the 1960s
  \begin{center}
    \includegraphics<1>[width=\textwidth]{pics/limiting_similarity}
    \includegraphics<2>[height=.8\textheight]{pics/finches}
    % \includegraphics<3>[width=.9\textwidth]{pics/hermoyian-2002}
  \end{center}
  % 
  \FrameText{\small%
    \only<1>{MacArthur and Levins 1967}%
    \only<2>{Lack 1947}%
    % \only<3>{Hermoyian et al. 2002}%
  }
\end{frame}

\begin{frame}
  % This sort of idea seems to have stronger roots in animal work -
  % it's less clear how these sorts of trait and resource axes should
  % apply to plants, who compete for relatively few resources.
  %
  % In particular traits that are thought to affect use of light and
  % other resources like specific leaf area don't map on to different
  % resources but to rates of uses of resources.
  %
  % So it's therefore very hard to see how to build a bridge between
  % our simple abstract models of competition and the types of traits
  % and species data that we tend to collect for plant communities.
  \includegraphics<1>[width=\textwidth]{figs/gaussian_competition_3}
  \includegraphics<2>[width=\textwidth]{figs/gaussian_competition_5}
  \includegraphics<3>[width=\textwidth]{figs/bridge_1}
\end{frame}

\begin{frame}
  % What I'm trying to do is to build part of this bridge by looking
  % at what competition looks like in models where competition is
  % implicit and emerges from the ingredients of the model, rather
  % than being explicitly coded into the model.  I'm going to do this
  % with two models: a simplest abstract model and a complicated
  % mechanistic model, representing ends of a modelling continuum.
  \LARGE
  What do competition kernels look like\\[.5ex]
  in models with \textbf{\color{orange-dk}implicit} competition?
  % The approach is simple minded: focus on invasion fitness of rare
  % mutants across trait space in the presence of a single resident.
  % Compute all the components of the MacArthur and Levins model:
  % fitness, mutant carrying capacity, mutant maximum growth rate and
  % resident density, to work out the per-capita effect of competition
  % of the resident on the invading species.
\end{frame}

\begin{frame}
  \begin{center}
    \includegraphics[height=.5\textheight]{pics/rockstar_logo}
    % TODO: paper heading from 1980 paper, the book cover, a big R*?
  \end{center}
  \FrameText{\small Tilman 1980, 1982}
  % Simplest model we could think of is Tilman's R* model, developed
  % through the 1980s, but still in widespread use.

  % The model has similar ingredients to the ML/LV model, but resource
  % regeneration does not happen instantaneously.

  % Species compete for and consume some number of resources -- at most
  % 'n' species can coexist on 'n' resources.
\end{frame}

\begin{frame}{R star with one resource}
  \begin{center}
    \vspace{-4ex}
    % With a single resource, a single species only can exist.
    % 
    % Here the species are characterised by their resource requirement
    % along the x axis -- this is the amount of resource required to
    % generate one offspring.
    % 
    % The species with the lowest requirement wins, so anything with a
    % lower requirement (to the left of the resident) can invade.
    % 
    % If you compute the competition kernel for this, it is asymmetric
    % and corresponds to our intuition here: species can invade where
    % they experience less competition than the resident species
    % exerts on itself (at 1)
    \includegraphics<1>[height=.8\textheight]{figs/rstar1_alpha_1}
    \includegraphics<2>[height=.8\textheight]{figs/rstar1_alpha_3}
  \end{center}
\end{frame}

\begin{frame}{R star with two resources}
  % Things start to get weirder with two resources.
  %
  % Imagine that the requirements for two essential resources
  % trade-off so that this maps to a single trait axis.
  % 
  % Species to the far left are specialists on resource A and species
  % to the far right are specialists on resource B.
  \begin{center}
    \vspace{-4ex}
    % This generates a weird and nondifferentiable kernel.
    \includegraphics<1>[height=.8\textheight]{figs/rstar2_alpha_1}
    \includegraphics<2>[height=.8\textheight]{figs/rstar2_alpha_3}
    \includegraphics<3>[height=.8\textheight]{figs/rstar2_alpha_4}
  \end{center}
  % This kernel is unsurprisingly also not constant with respect to
  % resident density (and the shape moves around as resident density
  % changes).

  % So this is weird: the most commonly used *implicit* model of
  % competition, often used in thought experiments about coexistance,
  % does not line up with the assumptions of the most commonly used
  % *explicit* models of competition.
  % 
  % Given that conclusions in the explicit model will depend on things
  % like the ratio of widths of the competition and carrying capacity
  % kernel (Dieckmann and Doebelli) or on the details of the spectral
  % composition of the kernsl (Leimar) these departures will at least
  % quantitatively change the conclusions of these models.
\end{frame}

\begin{frame}% {Linking to biological traits}
  % TODO: Uncertain if a slide heading through this (and the previous)
  % section are useful.

  % OK, but the R* model is still really abstract, and doesn't really
  % map onto traits that we see in nature (things like the LES affect
  % resource use but don't directly map onto it).

  % Ideally what *I'd* like to see is what these kernels look like in
  % trait space in a biological community.  But I'm not much of an
  % empiricist.

  % Instead, consider a mechanistic model where we model competition
  % for light, and incorportate information about trade-offs

  \begin{center}
    \includegraphics<1>[height=.8\textheight]{figs/tree_model}
  \end{center}
  \FrameText{\small Falster et al. 2010}
\end{frame}

\begin{frame}
  % Height at maturation is an obvious arms-race like trait in this
  % system -- the taller you are the more you shade out other
  % individuals, but the longer you delay reproduction.
  %
  % There's a lot of possible outcomes of this model, but here is one
  % - fitness tends to decrease with increasing height at maturation.
  %
  % However, the competition kernel goes the expected direction: the
  % shorter you are the more competition you face from the taller
  % species -- shorter species are able to invade in spite of intense
  % competition from the invader.
  %
  % This set of parameters actually leads to reasonable levels of
  % coexistance without any gaussian kernels or limiting similarity.
  \begin{center}
    \includegraphics<1>[height=.9\textheight]{figs/tree_hmat}
    \includegraphics<2>[width=\textwidth]{figs/tree_w_hmat}
    \includegraphics<3>[width=\textwidth]{figs/tree_alpha_hmat_2}
  \end{center}
\end{frame}  

\begin{frame}
  % 
  % Leaf mass per area is a core component of the leaf economic
  % spectrum, correlateing strongly with leaf nitrogen, photosynthetic
  % rate and leaf lifespan.  The larger your LMA the more conservative
  % your strategy is.
  %
  % TODO: fitness slide with nothing on except for the resident and
  % traits.
  %
  % This trait generates really odd looking fitness landscapes, partly
  % due to how *time* is being partitioned.
  %
  % The competition kernel underlying these fitness landscapes is very
  % rugged with peaks and troughs.  Here, species with a lower LMA
  % than the resident can invade, but for species with a higher LMA
  % than the resident the landscape is very lumpy.
  %
  % As the resident evolves towards the optimim, these bumps move
  % around so that some bumps drop quite low - these regions of low
  % competition allow invasion of other types.  This is a potential
  % mechanism of coexistance that falls outside the scope of
  % traditional limiting-similarity style models.
  \begin{center}
    \includegraphics<1>[width=\textwidth]{pics/lma}
    \includegraphics<2>[width=\textwidth]{figs/tree_w_lma}
    \includegraphics<3>[width=\textwidth]{figs/tree_alpha_lma_2}
  \end{center}
\end{frame}

\begin{frame}
  % I've attempted to look at how to link abstract models of
  % competition with traits that we actually have by looking at models
  % where competition emerges from the model.  So far, nothing looks
  % like the Gaussian kernels that we generally work with.
  %
  % This isn't meant as a criticism of the simple models -- there are
  % many resons to model competition and anything much more
  % complicated than Gaussian kernels quickly becomes intractable.
  % 
  % However, criticism of these models is as old as the models
  % themselves, especially by Peter Abrams.
  %
  % More concerning is that we don't have a strong idea about how
  % departures from the Gaussian kernel will impact models of
  % coexistance or of other models for which competition is a key
  % ingredient.  There is some quite theoretical work on this, but
  % it's still a long way removed from the really bizarre shapes that
  % I've seen.
  %
  % Ideally we can odentify features that mechanistic models, models
  % with implicit competition, and emprical data have in common and
  % make sure that our models that invoke explicit competition are
  % robust to these features.
  %
  % Measuring competition in mechanistic models has potential for
  % bridging theory and data.  Unexpected shapes and forms.  May help
  % with nonmanipulative estimation of competition intensity in the
  % field.  At least that's what I'd like.  It certainly seems
  % unlikely to be worse than the current crop of models.
  \includegraphics[width=\textwidth]{figs/bridge_1}
\end{frame}

\begin{frame}{Acknowledgements}
  \begin{itemize}
  \item Westoby Lab
  \item Stephen Hardy, Felix Lawrence, Conrad Sanderson\\
    (National ICT Australia)
  \item Science Industry Endowment Fund,\\Australian Research Council
  \end{itemize}
\end{frame}

\end{document}

%%% Local Variables: 
%%% mode: latex
%%% TeX-PDF-mode: t
%%% TeX-engine: xetex
%%% End: 

%  LocalWords:  Tilman Dieckmann Lande Leimar Theor Ispolatov Geritz
%  LocalWords:  dieckmann kisdi Jacobian rstar Lotka Volterra
